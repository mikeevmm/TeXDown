% Start of header.
\documentclass{article}
\usepackage{url}
\usepackage{amsmath}
\usepackage[utf8]{inputenc}
\usepackage{amssymb}
\usepackage[normalem]{ulem}
\usepackage{listings}
\usepackage{csquotes}
\usepackage{amsthm}

\newcommand{\constant}[0]{\frac{1}{4 \pi \epsilon_0}}
\newcommand{\withArgs}[4]{\frac{#1}{#4}*#2 \cdot #2}

\newtheorem{theorem0}{Pythagorean Theorem}
\newtheorem{theorem1}{Theorem}
\newtheorem{theorem2}{Name}
\newtheorem{theorem3}{Lemma}

\title{}
\author{Miguel Murça}
\date{November 10, 2016}
% End of header
% Start of body.

\begin{document}

\maketitle

\vspace{5mm}


\vspace{5mm}


\vspace{5mm}

\section{This is a section}

\begin{theorem0}
    A rectangle triangle of sides $a$, $b$ and $c$ must be
    such that
    $a^2 + b^2 = c^2$
\end{theorem0}

\subsection{This is a subsection}

\subsection*{This is an unnumbered subsection}

\subsubsection{Subsubsection}

\paragraph{Paragraph}

\subparagraph{Subparagraph}

\begin{theorem1}
    Unnamed theorem.
\end{theorem1}

\begin{theorem2}
    A named theorem
\end{theorem2}

\begin{theorem3}
    A lemma
\end{theorem3}

This is text,
and this is still the same paragraph.

\textbf{BOLD}

\emph{emphasis}
\emph{italics (same as emphasis)}

\underline{underline}

\textbf{Some \emph{sweet} \underline{nested} and I mean \underline{\emph{nested}} formats.}

\sout{strikeout}

\sout{strikeout w/ 2}

\begin{gather}
\textrm{I can haz numbering?}
\end{gather}

\lstinline[columns=fixed]$How about inline code?$

\begin{gather*}
e^x = x^e cont eq 1\\
eq 2
\end{gather*}

$ y = x + y $

$ \cos x $

\begin{lstlisting}[language=python,]
# SHOULD BE A COMMENT
This is some sweet code, right here.
\end{lstlisting}

\begin{lstlisting}[language=python,caption=This is code with a caption.]
myStr = 'example'
def longExample:
    foo = input()
    bar = raw_input()

    for i in range(1,input()):
        pass
    # Long
\end{lstlisting}

\rule{\textwidth}{0.4pt}

You can also type standard \LaTeX{} with nice things
such as the paragraph space or $\textrm{inline math: }x^2$.

\begin{itemize}
\item An
\item Unordered
\item List
\begin{itemize}
\item with
\item sub
\item items
\item and
\begin{itemize}
\item sub
\item items
\end{itemize}
\end{itemize}
\end{itemize}
\begin{enumerate}
\item A
\item Really
\begin{enumerate}
\item Sexy
\end{enumerate}
\item Ordered
\item List
\end{enumerate}

\vspace{5mm}

\begin{table}[h!tpb]
\begingroup
\setlength{\tabcolsep}{10pt}
\renewcommand{\arraystretch}{1.5}
\begin{tabular}{ |l|c|r|l| }
\hline
Tables & Are & Cool \\ \hline \hline
col 3 is & right-aligned & \$1600 \\ \hline
col 2 is & centered & \$12 \\ \hline
zebra stripes & are neat & \$1 \\ \hline
\end{tabular}
\label{table1}
\endgroup
\end{table}

\vspace{5mm}

\begin{table}[h!tpb]
\begingroup
\setlength{\tabcolsep}{10pt}
\renewcommand{\arraystretch}{1.5}
\begin{tabular}{ |l|l|l| }
\hline
\sout{Markdown} \LaTeX{}Down & Less & Pretty \\ \hline \hline
\emph{Still} & \lstinline[columns=fixed]$renders$ & \textbf{nicely} \\ \hline
1 & 2 & 3 \\ \hline
\end{tabular}
\label{table3}
\endgroup
\end{table}

\begin{table}[h!tpb]
\begingroup
\setlength{\tabcolsep}{10pt}
\renewcommand{\arraystretch}{1.5}
\begin{tabular}{ |l|l|l|l| }
\hline
Pretty & Markdown & Tables \\ \hline \hline
Can & also & have \\ \hline
\end{tabular}
\label{table2}
\caption{Captions, as evidenced!}
\endgroup
\end{table}

\vspace{5mm}

Miguel Murça, creator of TeXDown, said
\begin{displayquote}
Here is a quote,

made using TeXDown.
\end{displayquote}
Truly words of wisdom.
\end{document}
% End of body.